\chapter{A chapter with equations, chemical formula, figures and tables}
\label{chap_main}

The previous chapter \ref{chap_intro} contained only some text. Here some sample
equations, chemical formula, figures, and tables
are introduced.

\section{Section with equations and chemical formula}
\label{sec_eq_fig_tab}

Materials science is highly interdisciplinary. So you will have to write complex
equations some condensed-matter physicists use like:

\begin{equation}
\expval{\hat{O}}{\Psi_e} = \int \dotsc \int \Psi_e^{*}(x_1,\dotsc,x_N) \hat{O}
\Psi_e(x_1,\dotsc ,x_N)  dx_1 \dotsi dx_N
\quad .
\label{main_eq1}
\end{equation}

and,

\begin{equation}
\braket{\Psi_e} = \int \dotsc \int |\Psi_e (x_1,\dotsc,x_N)|^2  dx_1\dotsi dx_N = 1
\quad .
\label{main_eq2}
\end{equation}

Sometimes, you would also have to write chemical formula like chemists:

\begin{equation}
\centering
\ch{H3O_{aq}+ + 2 H2O <-> H5O2_{(aq)}+ + H2O <-> H7O3_{(aq)}+} \quad .
\label{main_eq3}
\end{equation}

With the appropriate tex packages defined, all this is possible.

\section{Section with tables and figures}
\label{sec_tab_figs}

Here is a sample table and figure. For the table, I've taken data from my
thesis \cite{Surendralal2020} and references \cite{Singh-miller2009, Sakong2018, Kittel1976, Salmeron1983}.
Notice how in table \ref{main_tab1}, I've rendered crystallographic directions using
the "miller" tex package.

\begin{table}[!tbh]
\begin{center}
  \begin{tabular}{llc}
  \toprule
       &  a [\AA{}]  &  $\Phi\mathrm{_{Pt\hkl(111)}}$ [eV]  \\
  \midrule
  PBE (this work) &  3.97  &    5.69  \\
  PBE (ref. \cite{Singh-miller2009}) & 3.99  &    5.69   \\
  RPBE (this work) &  3.99 &   5.50 \\
  RPBE (ref. \cite{Sakong2018})&  3.99 &   5.51 \\
  Experiments &  3.92 \cite{Kittel1976} &   6.08 $\pm$ 0.15 \cite{Salmeron1983} \\
  \bottomrule
  \end{tabular}
\end{center}
\caption[Calculated lattice parameter of Pt and work function of the Pt\hkl(1 1 1)
surface.]{The calculated lattice parameter of bulk Pt and the work function of the
Pt\hkl(1 1 1) surface using different exchange-correlation functionals compared to
experimental values.}
\label{main_tab1}
\end{table}

\begin{figure}[!tbh]
\centering
\includegraphics[scale=1., height=8cm]{\figpath/sinx.pdf}
\caption[A sinus curve]{A sinus curve to show how to include figures along with
captions.} %The text in square brackets is what is rendered in the list of figures.
\label{main_fig1}
\end{figure}

\section{Section with code}
\label{sec_code}

If you are in a computational field, you might need to include some code in your
thesis. The "minted" TeX package is useful for this. For example, some python code
can be inserted the following way:

\begin{minted}{python}
    import numpy as np
    import matplotlib.pylab as plt
\end{minted}

You could also insert inline code like so: \mintinline{python}{numpy}.
